\begin{frame}
    \frametitle{Conclusion}
    \begin{itemize}
        \item Accounting for sample selection in dyadic settings requires more involved methods than the standard \cite{heckman1979sample} approach.
        \item Difficulty arises through the incidental parameter problem in the selection equation.
        \item \cite{charbonneau2017multiple} estimator:
        \begin{itemize}
            \item Leads to estimates that are free of this problem in the first stage.
            \item However, it does not deliver estimates of fixed effects essential to employ Heckman approach.
        \end{itemize}
        \item To bypass this problem, we proposed:
        \begin{itemize}
            \item To get the FE through a Hybrid estimation and further apply the Heckman procedure after a transformation is made in the selection equation. \\
            $\xrightarrow{}$ it may not be well suited for sparse networks
            \item A modification to the \cite{kyriazidou1997estimation} estimator. \\
            $\xrightarrow{}$ no need for FE estimates.\\
            $\xrightarrow{}$ estimates are sensitive to chosen bandwidth. \\
            $\xrightarrow{}$ needs a variable that satisfies exclusion restriction
        \end{itemize} 
        \item Drawbacks: computational costs!
        \item Simulations confirm the theoretical predictions that biases are reduced.
    \end{itemize}
\end{frame}