\section{Conclusion} \label{conclusion}
In this study we showed that accounting for sample selection bias in dyadic data settings is not straightforward and requires more involved methods than the standard \cite{helpman2008estimating} two-step approach. The standard Heckman approach involves a first stage where the probability of observing an outcome is estimated through a probit model with dummys for fixed effects, and a second stage that estimates the effects of different variables on such outcome, taking into account the estimated probabilities in the first stage.

The motivation to study dyadic interactions came from the gravity models for international flows. We specifically consider the model proposed by \cite{helpman2008estimating}, since it explicitly takes into account how the sample selectivity arises when estimating how trade barriers affect trade flows, when considering in the estimation only countries with positive trade flows. Another key aspect in this model is the presence of multilateral resistance terms, that boils down to the inclusion of two-way fixed effects for both countries involved in the trade.

Even though the motivation emerged from the trade literature, the approaches outlined in this study can be applied to any dyadic interaction where sample selectivity might be present (given that the model satisfies the outlined assumptions throughout the study). More generally, the first stage estimations, which outline the probability of observing a pairwise interaction that generates an outcome (in the case above, the probability of trading with each other), can be seen as an estimation of a network formation model, as highlighted in \cite{jochmans2018semiparametric}. The presence of the two-way fixed effects also indicates a close relationship to the $\beta$-models in the networks literature.

The difficulty in correcting for sample selectivity in such models arises as the two-way fixed effects are incorporated in the first stage estimation. These effects leads to the incidental parameter problem (\cite{neyman1948consistent}), and the estimates of structural parameters obtained through the standard Probit are asymptotically biased (\cite{fernandez2016individual}).

We showed more formally how this problem arises, and proposed the conditional logit estimates of \cite{charbonneau2017multiple} to estimate the structural parameters of the first stage equation. Such method relies on a clever way of differencing out both fixed effects in the equation over quadruples of dyads. The downside of this method is that it evidently does not provide estimates of the fixed effects, and therefore, the estimation of the inverse Mills-ratio for the traditional Heckman approach are not readily available.

To bypass this problem, we proposed two approaches based on existing methods, but providing the suitable modifications to this setting of dyadic regressions. The first method, denoted by Hybrid, relies on retrieving the estimates of the fixed effects through the unconditional log-likelihood, restricting the structural parameters to be the same as the estimates obtained by the Charbonneau estimates. Once the fixed effects estimates are collected, it is possible to transform the variables, as proposed by \cite{lee1983generalized} such that the Heckman two-step approach can be employed even when the errors in the selection equation are logistically distributed. The advantages of this approach is that there is no need for a variable that satisfies an exclusion restriction, since the sample selection effects have a defined functional form in the observation equation, guaranteeing the identification of its structural parameters. The downsides of this method are that (i) one still needs to obtain estimates for the fixed effects in the first stage equation, pointing out that this approach may not be suitable for sparse networks, and that (ii) the distribution of the errors need to be specified.

The second method is based on \cite{kyriazidou1997estimation}, and involves differencing the second stage equation over quadruples of dyads. The form of differencing guarantees that the fixed effects in the second stage equation are completely differenced out, however it does not guarantee that the sample selectivity effects are completely differenced out as well. To overcome this problem, the main idea is then that weights are applied to the transformed observations such that a higher weight is attributed to quadruples for which the differences in the sample selectivity effects are smaller. Such differences in sample selectivity effects take into account the estimates provided by Charbonneau of the first stage. The benefits of this method is that there is no need to estimate any of the fixed effects in both equations, and to specify the distribution of the errors in the observation equation. The deficiencies of this approach are that the estimates are sensitive to the choices of bandwidths $h$, as shown in the application and in the Monte Carlo simulations. Also, the proposed method to choose an optimal bandwidth seems to not show improvements. One aspect of both methods is that they are computationally costly, since they boil down to combinatoric problems over quadruples of dyads that must satisfy some given restrictions.

Our Monte Carlo simulation exercise confirms the theoretical predictions that under the normality assumption of both error terms, the estimators of the first stage equation given by the Probit model with fixed effects are biased. On top of that, such biases are carried over to the estimations of the second stage equation, given by the standard Heckman approach. The exercise also demonstrates that,as predicted by the theory, the proposed methods (under their respective assumptions) deliver reductions in the biases the estimators of both equations.

However, we must point out that in our empirical application to the gravity model for international trade flows, it is not straightforward to draw conclusions from the obtained estimates from the different approaches for some variables. One of the potential sources for this limitation is that the method based on \cite{kyriazidou1997estimation} requires a variable that satisfies an exclusion restriction. In our application it seems to be the case that the proposed variable does not satisfy the requirement that it should not affect the outcome of the second stage equation.

Finally, further research topics related to this study are: (i) obtaining the asymptotic properties of both proposed estimators, (ii) allow for a setting with several time periods, (iii) investigate whether a better method for choosing the bandwidth $h$ in the second approach can be obtained and (iv) study the possibility of extending the methods such that it is possible to be employed in networks where transitivy is present (which involves relaxing assumptions related to errors being i.i.d.).

