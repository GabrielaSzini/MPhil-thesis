\section{The incidental parameter problem} \label{section_incidental_parameters}
Even though \cite{helpman2008estimating} directly estimated Equation (\ref{eq:est_MHR_sampleselection2_probit}) by maximum likelihood, it is well known in the literature that the fixed effects estimators for nonlinear panel data models suffers from the incidental parameter problem (\cite{neyman1948consistent}), yielding asymptotically biased estimates of the parameters.

As highlighted by \cite{arellano2007understanding} in a standard panel data regression with one way fixed effects and dimensions $i=1,...N$ and $t=1,...T$, if $T$ is fixed and $N \xrightarrow{} \infty$, there will be an estimation error in the estimates of the fixed effects, as only a finite number $T$ of observations are available to estimate each fixed effect. As we allow for the fixed effects to be correlated with the exogenous regressors (and its distribution is left unspecified), this estimation error contaminates the estimates of the other parameters as well, as they are not informationally orthogonal. For large enough $T$, this bias should be small. However, even under $T \xrightarrow{} \infty$ and  $N \xrightarrow{} \infty$, the fixed effects estimator will be asymptotically biased, leading to incorrect inference over the parameters and the average partial effects.

The same argument holds for our framework of a dyadic regression with two-way fixed effects. In our panel data model, we have 3 dimensions: $i=1,..N$, $j=1,...N$ and $t=1,...T$. Thus, the first two dimensions grow at rate $N$, and the latter at rate $T$. As most of the available datasets in the trade literature have the dimension $T$ fixed, we will consider asymptotic results such that $N \xrightarrow{} \infty$ and $T$ is fixed. 

Note as well that for each new country in the dataset, the number of observations is increased by $2(N-1)T$. Moreover, for each fixed effect in Equation (\ref{eq:model3}) there are $(N-1)T$ observations available for their estimation. 

We will now use results shown by \cite{fernandez2016individual} to demonstrate how the incidental parameter problem arises in our framework, delivering consistent but asymptotic biased estimators, keeping in mind that as $N \xrightarrow{} \infty$ both dimensions $i$ and $j$ go to infinity and also the number of observations $N(N-1)T$ go to infinity.

Given the dataset of $N(N-1)T$ observations $\{ (y_{2,ij,t}, x_{2,ij,t}')' : 1 \leq i \leq N, 1 \leq j \leq N, 1 \leq t \leq T, i \neq j \}$ with $y_{2,ij,t} = \mathbbm{1} (y_{2,ij,t}^{**} > 0)$, and $y_{2,ij,t}^{**}$ specified by Equation (\ref{eq:model3}) where $\eta_{ij,t}^{*}$ is i.i.d., we have that $y_{2,ij,t}$ is generated by the process:
\begin{align*}
    y_{2,ij,t} \rvert x_{2,ij,t}, \xi^*, \zeta^*, \beta_2^* \sim f_{Y_2}(\cdot \rvert x_{2,ij,t}, \xi^*, \zeta^*, \beta_2^*)
\end{align*}

\noindent where: $\xi^* = (\xi_1^*, ... \xi_N^*)$, $\zeta^* = (\zeta_1^*, ... \zeta_N^*)$, $f_{Y_2}$ is a known probability function and $\xi^*_i, \zeta^*_j$ are the unobserved fixed effects.
Note here that this approach is semi-parametric in the sense that is does not specify the distribution of the fixed effects or their relationship with the explanatory variables.

We can further model the conditional distribution of $y_{2,ij,t}$ using a single-index specification with fixed effects, since it is a binary response model: 
\begin{align}
   f_{Y_2} ( y_{2,ij,t} \rvert  x_{2,ij,t}, \xi^*, \zeta^*, \beta_2^*) &= F(x_{2,ij,t}'{\beta_2^*}  +\xi_{i}^*+\zeta_{j}^*)^{y_{2,ij,t}} \nonumber\\ 
   &\times [1 - F(x_{2,ij,t}'{\beta_2^*}  +\xi_{i}^*+\zeta_{j}^*)]^{1-y_{2,ij,t}}, 
   \label{eq:fy}
\end{align}

\noindent where, clearly $y_{2,ij,t}  \in \{0,1\}$ and $F$ is a cumulative distribution function, defined to be a standard normal according to Assumption \ref{assumption_heckman_3}.

We can then collect all the fixed effects to be estimated in the vector $\omega_{NN}^* = {(\xi_1^*, ... \xi_N^*, \zeta_1^*, ... \zeta_N^*)}'$, which can be seen as a nuisance parameter vector. Then, the true values of the parameters, denoted by $\beta_{2,0}^*$ and $\omega_{NN,0}^* = {(\xi_{1,0}^* , ... \xi_{N,0}^* , \zeta_{1,0}^* , ... \zeta_{N,0}^*)}'$ are the solution to the population conditional maximum likelihood maximization:
\begin{align}
    \max_{(\beta_2^*, \omega_{NN}^*) \in \mathbb{R}^{\text{dim} \beta^* + \text{dim} \omega_{NN}^*}} \mathbb{E}_\omega [\mathcal{L} (\beta_2^*, \omega_{NN}^*)]
    \label{eq:val1}
\end{align}

\noindent with 
\begin{align} \label{eq:unconditional}
    &\mathcal{L} (\beta_2^*, \omega_{NN}^*) \\ &= (N(N-1)T)^{-1} \Big\{ \sum_{i=1}^{N}\sum_{j\neq i}\sum_{t=1}^T \log f_{Y_2} ( y_{2,ij,t} \rvert  x_{2,ij,t}, \xi^*, \zeta^*, \beta_2^*) - b(\iota'_{NN} \omega_{NN}^*)^2/2 \Big\} \nonumber
\end{align}

\noindent where $\mathbb{E}_\omega$ denotes the expectation with respect to the distribution of the data conditional on the unobserved effects and strictly exogenous variables, $b>0$ is an arbitrary constant, $\iota_{NN} = (1_N', - 1_N')'$ and $1_N$ denotes a vector of ones of dimension $N$.

The second term of $\mathcal{L}$ relates to a penalty that imposes a normalization to identify the fixed effects in models with two-way fixed effects that enter in the log-likelihood function as $\xi_i^* + \zeta_j^*$. To be more specific, in this case, adding a constant to all $\xi_i^*$ and subtracting the same constant from all $\zeta_j^*$ would not change $\xi_i^* + \zeta_j^*$. Thus, without this normalization, the parameters $\xi_i^*$ and $\zeta_j^*$ are not identifiable.

To estimate the parameters, we solve the sample analogue of the following equation:
\begin{align}
    \max_{(\beta_2^*, \omega_{NN}^*) \in \mathbb{R}^{\text{dim} \beta_2^* + \text{dim} \omega_{NN}^*}} \mathcal{L} (\beta_2^*, \omega_{NN}^*)
    \label{eq:val1}
\end{align}

In order to analyze the statistical properties of $\beta_2^*$, we first concentrate out the nuisance parameters $\omega_{NN}^*$, such that for given $\beta_2^*$, the optimal $\hat{\omega}_{NN}^*(\beta_2^*)$ is:
\begin{align}
 \hat{\omega}_{NN}^*(\beta_2^*) = \argmax_{\omega_{NN}^* \in \mathbb{R}^{\text{dim} \omega_{NN}^*}} \mathcal{L} (\beta_2^*, \omega_{NN}^*)   
\end{align}

Thus, the fixed effects estimator of $\beta_2^*$ and $\omega_{NN}^*$ are, by plugging in the previous expression for $\hat{\omega}_{NN}^*(\beta_2^*)$:
\begin{align} \label{eq:taylor}
   \hat{\beta}_2^* = \argmax_{\beta_2^* \in \mathbb{R}^{\text{dim} \beta_2^*}} \mathcal{L} (\beta_2^*, \hat{\omega}_{NN}^*(\beta_2^*))
\end{align}
\begin{align} \label{eq:omega}
    \hat{\omega}_{NN}^*(\beta_2^*) = \hat{\omega}_{NN}^*(\hat{\beta}_2^*)
\end{align}

The source of the problem is that the dimension of the nuisance parameters $\omega_{NN}$ increases with the sample size under asymptotic approximations where $N \xrightarrow{} \infty$. To further describe the incidental parameter problem, denote:
$$\Bar{\beta}^*_2 = \argmax_{\beta^*_2 \in \mathbb{R}^{\text{dim} \beta^*_2}} \mathbb{E}_\omega \Big[\mathcal{L} (\beta^*_2, \hat{\omega}_{NN}^*(\beta^*_2)) \Big]$$

Using an asymptotic expansion for smooth likelihoods under appropriate regularity conditions, provided by \cite{fernandez2016individual}, we have that:
\begin{align} \label{eq:val2}
    \Bar{\beta}_2^* = \beta_{2,0}^* + \frac{\Bar{B}_\infty}{(N-1)T} + \frac{\Bar{D}_\infty}{(N-1)T} + \text{o}_P(((N-1)T)^{-1})
\end{align}

For some constants $\Bar{B}_\infty$ and $\Bar{D}_\infty$. The derivation for this expression can be found in the Appendix of \cite{fernandez2016individual}. As explained by the authors, the expansion is obtained by first taking a first-order Taylor expansion of the Equation (\ref{eq:taylor}) around the true value $\beta_{2,0}^*$, as it is usually done to obtain the asymptotic properties of such estimator. Then, one should additionally take a second-order Taylor expansion of the obtained term $\frac{\partial \mathcal{L} (\beta_{2,0}^*, \hat{\omega}^*_{NN})}{\partial_{\beta_2^*}}$ around the true values of the nuisance terms. Intuitively, this second step demonstrates how the estimates of the fixed effects affect the estimates of the structural parameter $\beta_{2}^*$. We provide a more formalized form of this argument in Appendix \ref{derivation_expansion}. To obtain the exact form of the expressions $\Bar{B}_\infty$ and $\Bar{D}_\infty$ a quite involved derivation is needed. However, this is not the focus of our study, since we show later that there are other possibilities to correct for the asymptotic bias generated by these terms other than deriving the biases themselves. 

Moreover, by the properties of the maximum likelihood estimator we have that, under regularity conditions:
\begin{align}
    \sqrt{N(N-1)T} (\hat{\beta}_2^* - \Bar{\beta}_2^*) \xrightarrow{d} N(0, \Bar{V_B}_\infty)
\end{align}

For some $\Bar{V_B}_\infty$. By substituting the expression for $\beta_{2,0}^*$ obtained in Equation (\ref{eq:val2}), we obtain that, by Slutsky's theorem:
\begin{align} \label{eq:val3}
    &\sqrt{N(N-1)T} (\hat{\beta}_2^* - \beta_{2,0}^*) \\ &=\sqrt{N(N-1)T} ( \hat{\beta}_2^* - \Bar{\beta}_{2}^*) \nonumber \\ 
    &+ \sqrt{N(N-1)T} \Big( \frac{\Bar{B}_\infty}{(N-1)T} + \frac{\Bar{D}_\infty}{(N-1)T} + \text{o}_P(((N-1)T)^{-1}) \Big) \nonumber\\
    &\xrightarrow{d} N\Big( \frac{\Bar{B}_\infty}{\sqrt{T}} + \frac{\Bar{D}_\infty}{\sqrt{T}}, \Bar{V_B}_\infty\Big) \nonumber
\end{align}

We can see from Equation (\ref{eq:val2}) that, as $N \xrightarrow{} \infty$, $\hat{\beta}_2^* \xrightarrow{p} \beta_{2,0}^*$ ($\beta_{2,0}^*$ being the true value of the parameter), thus, the estimates of $\beta_{2,0}^*$ are consistent. However, from Equation (\ref{eq:val3}) we see that when $T$ is fixed, the estimates converge to a distribution that is not centered at zero, which leads to incorrect asymptotic confidence intervals. This demonstrates the incidental parameters problem, that boils down to an asymptotic bias in the estimates of $\beta_{2,0}^*$. This asymptotic bias arises as the order of the bias is higher than the inverse of the sample size because of the small rate of convergence of the fixed effects.

\begin{conjecture}
    The asymptotic bias in the estimates of $\beta_{2,0}^*$ carries over both to the estimates of the fixed effects, $\hat{\omega}_{NN}^*$ and to the estimates of the inverse Mills-ratio, $\hat{\lambda}_{ij,t}$.
\end{conjecture}

This conjecture follows from Theorem B.1. in the appendix of \cite{fernandez2016individual}, where it is shown that the asymptotic expansion of $\hat{\omega}_{NN}^*$ around $\omega_{NN,0}$ depends on $(\beta_2^* - \beta_{2,0}^*)$. Moreover, as we see from Equation (\ref{eq:omega}), the estimates of the fixed effects are a function of the estimates of $\beta_2^*$. Therefore, it is expected that given an asymptotically biased estimator $\hat{\beta}_2^*$, the estimators of the fixed effects will also converge to a limiting distribution not centered around its true values. Also, as mentioned by \cite{fernandez2016individual}, its rate of convergence is slower than that of the structural parameter, $\sqrt{(N-1)T}$. Thus, we expect that, under fixed T:
\begin{align} \label{eq:val4}
\sqrt{(N-1)T} (\hat{\omega}_{NN}^* - \omega_{NN,0}) \xrightarrow{d} N\Big( {\Bar{A}_\infty} , \Bar{V_A}_\infty\Big).
\end{align}

For some constant $\Bar{A}_\infty$. Remembering that $\hat{\lambda}_{ij,t}$ is a function of the estimated fixed effects and of the estimated structural parameters, if the rates of convergence of both random variables would be the same, one could straightforwardly apply the Delta method to determine the distribution of $\hat{\lambda}_{ij,t}$. 

In our case, as the fixed effects converge at a slower rate, we expect that the joint distribution will only converge at the slower rate. On this note, if we scale the random variable $(\hat{\beta}_2^* - \beta_{2,0}^*)$ by this rate it is expected that it converges to a constant denoted by $\Bar{C}_\infty$. Thus:
\begin{align*}
    \sqrt{(N-1)T} \begin{pmatrix}
        \hat{\beta}_2^* - \beta_{2,0}^* \\
        \hat{\omega}_{NN}^* - \omega_{NN,0} 
    \end{pmatrix}
    \xrightarrow{d} N \Big( 
    \begin{pmatrix}
        \Bar{C}_\infty \\
        \Bar{A}_\infty 
    \end{pmatrix} ,
    \begin{pmatrix}
        0 & 0 \\
        0 & \Bar{V_A}_\infty
    \end{pmatrix} \Big)
\end{align*}

Having defined this joint distribution, one can apply the Delta method easily, verifying that the limiting distribution of $\hat{\lambda}_{ij,t}$ is asymptotically biased even if the constant $\Bar{C}_\infty$ is zero. In case the constant is not zero, the only possibility for it being asymptotically unbiased would be if the biases from the fixed effects and from the structural parameters cancel out. However, there is no reason to believe that this should be the case. 

Under this conjecture, it is noteworthy that the conditions of Theorem \ref{theorem_heckman} do not hold. Therefore, if the selection equation is estimated by probit in this case, it is likely that the estimators of $\beta_1$ in Equation (\ref{eq:model1}) will also be asymptotically biased. A more formalized version of this conjecture is left for further research. For now, we aim to explore the finite sample properties of such estimator through Monte Carlo simulations.

To start addressing this issue, one can start looking at a possible asymptotically unbiased estimator for the structural parameter $\beta_2^*$. \cite{fernandez2016individual} tackles this problem by deriving asymptotic corrections, based on defining analytically the values of $\Bar{B}_\infty$, $\Bar{D}_\infty$ and $\Bar{V_B}_\infty$. Other studies propose different solutions for the incidental parameter problem in nonlinear panel data models. 
To name a few, \cite{jochmans2019modified} proposes modifications of the profile likelihood in models for dyadic interactions, covering the so called $\beta$-models in the framework of network formation, and delivering asymptotic unbiased estimators; \cite{graham2017econometric} provides an estimator for undirected dyadic link formation in network models, which takes into account the degree heterogeneity of agents (the fixed effects in our approach) through a tetrad logit (TL) estimator that conditions on sufficient statistics for the degree heterogeneity; and \cite{dzemski2019empirical} also provides analytical asymptotic corrections for estimating a model of directed dyadic link formation.

We propose in this study, instead of relying on approximate asymptotic corrections for such biases, to estimate the parameters through a conditional logit approach proposed by \cite{charbonneau2017multiple}. Such approach differences away both fixed effects. The advantage of the estimator proposed by \cite{charbonneau2017multiple} is that the analytical form of the biases does not need to be specified, while still delivering consistent and unbiased estimates of the parameters. 

