\section{Introduction} \label{introduction}
\cite{tinbergen1962shaping} established gravity equations, which have been widely used for estimating models of international trade, migration, equity and FDI flows. For instance, in the literature of international trade flows, it is used to infer the effects of institutions such as customs unions, exchange rate mechanisms, ethnic ties, linguistic identity and international boarders on trade flows. 

Even decades after its first appearance in trade models, there is still a substantial lack of understanding on features of its estimation, and econometric methods that considers some potential issues only emerged recently in the literature. The aim of this paper is to provide an ample and deep discussion about different estimation issues in gravity models focusing on its dyadic structure, highlighting and describing in detail proposed estimators that take it into account. 

A great deal of applications rely on the specification of the gravity model proposed by \cite{helpman2008estimating}, due to its tractability and the fact that it demonstrates explicitly how the sample selection problem plays a role in the macroeconomic model. Up to this seminal paper, most studies only considered observations with positive trade flows when estimating such model without imposing any special treatment on it. However, the authors point out that one should account for the self-selection of firms into export markets and their impact on trade volumes. In other words, one should account for sample selection in the estimation. To do so, the authors first relax some assumptions in the theoretical foundation of the gravity model proposed by \cite{anderson2003gravity}, then they propose a parametric estimation of a linear model controlling for the sample selection through the inverse Mills-Ratio approach (\cite{heckman1979sample}). 

However, we show that this methodology suffers from the incidental parameter problem, \cite{neyman1948consistent}. This well-known problem relates to the fact that, as demonstrated by \cite{fernandez2016individual}, the fixed effects estimates in nonlinear models are asymptotically biased, even when both dimensions of a panel dataset tend to infinity. The model proposed by \cite{helpman2008estimating} undergoes such issue, once it accounts for fixed effects for both importer and exporter, which are known in the literature as the multilateral resistance terms.

In order to analyze such problems, and discuss possible solutions to it, we further simplify the estimated equations by \cite{helpman2008estimating}. This simplification highlights the fact that such a framework is applicable to any dyadic data. As defined by \cite{graham2020dyadic}, dyadic data reflects situations where the outcome of interest is determined by pairwise interactions between units. More specifically, the first stage of the estimation - that relies on estimating the probabilities that two units (nodes) interact with each other (that countries trade in the trade application) - delineates a specification for network formation. One particularity of dyadic settings is that as the number of individuals increase, both dimensions of the pseudo panel increase.

The introduction of two-way fixed effects requires that in the first stage of the approach, a more involved estimator needs to be employed to difference out such fixed effects, delivering asymptotically unbiased and consistent estimates for the structural parameters. A clever approach is proposed by \cite{charbonneau2017multiple} and employed in this study. This approach is based on a conditional likelihood estimator.

Once such estimates are obtained, the standard Heckman approach for correcting for sample selection in the observation equation requires that (i) estimates of the fixed effects itself in the first stage (selection equation) are provided such that the predicted probabilities are obtained; and (ii) a functional form for the inverse Mills-ratio is provided, which generally follows from both errors being normally distributed. The \cite{charbonneau2017multiple} estimation results in additional problems related to both items: it does not deliver estimates for the fixed effects itself, and it relies on the assumption that the errors of the selection equation are logistically distributed.

We propose two different approaches to tackle such obstacles. The first is to retrive the estimates of the fixed effects through the traditional unconstrained MLE, by setting the structural parameters to be equal to the estimates provided by \cite{charbonneau2017multiple}. This method is denoted by the hybrid approach by \cite{martin2018bls}. Once the fixed effects estimates are obtained, one can then calculate the predicted probabilities. Moreover, a transformation of the variables proposed by \cite{lee1983generalized} guarantees that the traditional two-step approach given by \cite{heckman1979sample} can be employed, even if the error term of the selection equation is logistically distributed.

A second approach is developed by \cite{kyriazidou1997estimation}. It relies on a weighted least squares estimator based on the idea of differencing out the sample selection effects in the observation equation (the second stage equation of \cite{heckman1979sample}), and, in the trade application, in the estimates for the volumes traded amongst countries. The advantage of this method is that, as we will demonstrate later, it does not require estimates of the fixed effects in the first stage equation. While the original framework in \cite{kyriazidou1997estimation} is a standard panel data model, where equations are differenced over the time dimension, we propose modifications of this approach to accomodate dyadic interactions, where equations are differenced over combinations of dyads. Such modification delivers an approach that can be employed even when the exogenous variables of the observation equation are invariant over time, varying only over dyads.

Even though the proposed approaches already exist in the literature separately (apart from our modification to the estimator proposed by \cite{kyriazidou1997estimation}), this study adds to the literature in that it combines those separate methods to account for sample selection in dyadic structures. Up to our knowledge, although the literature for sample selection corrections is abundant for standard panel data models, it is insufficient for dyadic datasets, and consequently for cases related to network formation.

The structure of this paper is as follows: Section \ref{gravity_model} presents the baseline gravity model by \cite{helpman2008estimating}, Section \ref{estimating_helpman} presents the estimation strategy employed by the authors, Section \ref{model} delineates our simplification of the model, highlighting its dyadic structure and its aspects related to network formation, Section \ref{section_heckman} provides the standard two-stage Heckman approach for this model, Section \ref{section_incidental_parameters} refers to a more formalized discussion of the incidental parameters problem, Section \ref{section_charbonneau} discusses the approach by \cite{charbonneau2017multiple}, Section \ref{first_approach} shows our first proposed approach to correct for sample selectivity in the observation equation, Section \ref{sample_selection_alternative} shows the second proposed approach, Section \ref{simulations} provides our Monte Carlo simulations results for the standard Heckman approach and the two new methods proposed for a set of different designs, Section \ref{application} provides an application to the estimation of the gravity model for international trade flows, and Section \ref{conclusion} concludes.


